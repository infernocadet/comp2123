\documentclass[12pt]{article}

% UPDATE THESE PARAMETERS

\newcommand{\sid}{510435765}
\newcommand{\assignment}{1}


\NeedsTeXFormat{LaTeX2e}
\usepackage[osf]{mathpazo}
\usepackage[svgnames]{xcolor}
\usepackage[T1]{fontenc}
\usepackage{amsmath,amsthm,amsfonts,amssymb,mathtools}
\usepackage{hyperref,url}
\usepackage[margin=2.7cm,a4paper]{geometry}
\usepackage{tasks}
\usepackage{xstring}
\usepackage[tikz]{mdframed}
\usepackage{environ}
\usepackage{etoolbox}
\usepackage{fourier-orns}
\usepackage{kvoptions}
\usepackage[]{units}
\usepackage{url}%
\usepackage[normal]{subfigure}

% math config

\DeclarePairedDelimiter\ceil{\lceil}{\rceil}
\DeclarePairedDelimiter\abs{\lvert}{\rvert}
\DeclarePairedDelimiter\set{\{}{\}}


% headers

\usepackage{fancyhdr}
\addtolength{\headheight}{2.5pt}
\pagestyle{fancy}
\fancyhead{} 
\fancyhead[L]{\sc comp2123} % Replace with comp2123 or comp2823
\renewcommand{\headrulewidth}{0.75pt}

% update these headers
\fancyhead[C]{\sc SID: \sid}
\fancyhead[R]{Assignment \assignment}


% pseudo-code typesetting configuration
\usepackage[tikz]{mdframed}
\usepackage[noend]{algpseudocode}

\newenvironment{pseudocode}
{
  \mdframed
  \algorithmic[1]
}
{
  \endalgorithmic
  \endmdframed
}

% problem 

\newcommand{\problem}[1]{\noindent \textbf{Problem #1.}}


\begin{document}

\problem{1}  Consider the following snippet of pseudocode that takes an array of $n$ integers. 

\begin{pseudocode}
    \Function{algorithm}{$a$}:
    \State $n \gets \text{length of } A$
    \State $num\_matches \gets 0$
    
    \For{$i \gets 0 \textbf{ to } n-1$}
        \For{$j \gets i+1 \textbf{ to } n-1$}
            \If{$A[i] = A[j]$}
                \State $num\_matches \gets num\_matches + 1$
            \EndIf
        \EndFor
    \EndFor
    \State \Return $num\_matches$
    \EndFunction
\end{pseudocode}

Your task is to: \\

\textit{ a) upperbound the running time of the algorithm in terms of $n$ using $O$-notation.}

The bulk of this algorithm comprises of a nested for-loop. The outer loop uses a pointer $i$ which runs from $0$ to $n-1$ where $n$ is the length of the array. The inner loop uses a pointer $j$ which runs from $i+1$ to $n-1$. Hence the inner loop runs $n-i-1$ times for each iteration of the outer loop.
The worst-case scenario is when we have to process through the full size of the array, i.e. $n$ times.
We can sum the series of how many times the inner loop runs by calculating the series of how many times \texttt{line 5} runs for each value of $i$ in \texttt{line 4}. This gives us:

$$\sum_{i=0}^{n-1} (n-i-1) = \sum_{i=0}^{n-1} (n-1) - \sum_{i=0}^{n-1} (i)$$

$$= n(n-1) - \frac{(n-1)n}{2}$$

$$= \frac{n(n-1)}{2}$$

$$=O(\frac{1}{2}n^{2} + \frac{1}{2}n)$$

Considering lines 2, 3, 6, 7 and 8 all run in $O(1)$ time \textit{(as comparisons and assignments are in constant time)}, we can drop them and take the most significant order of $n$, hence this algorithm runs in $O(n^2)$ in the worst-case scenario. \\

\textit{b) lowerbound the running time of the algorithm in terms of $n$ using $\Omega$ notation.}

Assume for simplicity that $n$ is even. To lowerbound the running time, we consider only the last $n/2$ integers of the array. Since this is part of the full execution, analysing this part gives us a lower bound on the total running time. This gives us:

$$\sum_{i=\frac{n}{2} + 1}^{n-1} (n-i-1) = \sum_{i=\frac{n}{2} + 1}^{n-1} (n-1) - \sum_{i=\frac{n}{2} + 1}^{n-1} (i)$$

Using the arithmetic series:

$$=\frac{n}{4}(2n-2) - \frac{n}{4}(\frac{n}{2} + n)$$

$$=\frac{n^2-2n}{2} - \frac{n^2+2n}{8}$$

$$=\frac{3n^2-10n}{8}$$

Although the number of iterations are reduced in this modified algorithm, the growth rate of the number of operations with respect to $n$ is still quadratic. This gives us a lowerbound of $\Omega(n^2)$. \\ 

\newpage
\problem{2}  Consider a stack where each element stores an integer value. We want to extend the stack ADT we saw in the lectures with the additional operations listed below. The operations should run in O(1) time and the running time of the other stack operations should remain the same as those of a regular stack:

\texttt{Count()} returns the number of items on the stack

\texttt{Sum()} returns the sum of all elements on the stack

\texttt{Average()} returns the average of all the elements on the stack
\\ 

\textit{For each operation, describe their implementation in English, argue the correctness and the running time.}

\subsection*{Standard push() and pop() operations} 
We can add two attributes to our Stack ADT Class, \texttt{self.\_size} and \texttt{self.\_sum}, which initially both are set to $0$. Assume that our Stack ADT retains the standard \texttt{push()} and \texttt{pop()} methods as shown in the lectures, which both have a time complexity of $O(1)$ constant time.\\

In our insertion methods, every time we \texttt{push()} an element $e$, we increment \texttt{self.\_size} by $1$ and add the value of $e$ to \texttt{self.\_sum}. \\

\begin{algorithmic}[1]
\Function{newpush}{$self, e$}
    \State $self.push(e)$
    \State $self.\_size \gets self.\_size + 1$ \Comment{Increments size of stack}
    \State $self.\_sum \gets self.\_sum + e$ \Comment{Increases sum of total elements in stack}
\EndFunction
\end{algorithmic}


The \texttt{newpush()} function is modified to increment \texttt{self.\_size} by 1 ($O(1)$ time) and and increment \texttt{self.\_sum} by $e$ ($O(1)$ time). As shown in lectures \texttt{push()} takes $O(1)$ time, hence \texttt{newpush()} takes $O(1)$ time.\\

In our deletion methods, every time we \texttt{pop()} an element, we decrement \texttt{self.\_size} by $1$ and decrease \texttt{self.\_sum} by the value of the element. \\

\begin{algorithmic}[1]
\Function{newpop}{$self$} \Return{int}
    \If{$self.is\_empty()$} \Comment{Checks if stack is empty}
        \State \Return \textbf{None}
    \EndIf
    \State $e \gets self.pop()$
    \State $self.\_size \gets self.\_size - 1$
    \State $self.\_sum \gets self.\_sum - e$
    \State \Return $e$
\EndFunction
\end{algorithmic}


The \texttt{newpop()} function checks if the stack is empty ($O(1)$ time, and it returns None), else it will pop the topmost value and assign it to $e$ ($O(1)$ time), and decrement the \texttt{self.\_size} by $1$ ($O(1)$ time) and decrement \texttt{self.\_sum} by $e$ ($O(1)$ time). As shown in lectures \texttt{pop()} takes $O(1)$ time, hence \texttt{newpop()} takes $O(1)$ time. \\

These functions both maintain a running time of $O(1)$.\\

These new functions however do take up minimally more \textit{constant space} $O(1)$ as only two additional integer variables are added regardless of the size of the stack.\\

To prove correctness, we initiate loop invariants to make sure that size and sum accurately track the number of elements and sum of elements. 

\begin{itemize}
    \item When there is nothing in the stack: \texttt{self.\_size} and \texttt{self.\_sum} are equal to $0$, which maintains the invariant.
    \item When an integer is pushed into the stack using \texttt{newpush()}: the size of the stack (\texttt{self.\_size}) is incremented by $1$, and the sum of the stack (\texttt{self.\_sum}) is increased by the value of the element. \texttt{self.\_size} and \texttt{self.\_sum} correctly track the number of integers and sum of elements in this stack.
    \item  When an integer is popped out of the stack using \texttt{newpop()}: the size of the stack (\texttt{self.\_size}) is decremented by $1$, and the sum of the stack (\texttt{self.\_sum}) is decreased by the value of the element. \texttt{self.\_size} and \texttt{self.\_sum} correctly track the number of integers and sum of elements in this stack.
\end{itemize}

Hence the size and sum variables correctly track the number of elements currently in the stack and the total sum of these elements.

\subsection*{count(), sum() \& average()} 

Our \texttt{count()} function will return the size of the stack. \\

\textbf{Time complexity}: This operation runs in $O(1)$ time as it returns a pre-computed value.\\

\textbf{Correctness}: This operation returns the value of \texttt{self.\_size} which is incremented or decremented every time an integer is pushed or popped in and out of the stack, respectfully. The invariant \texttt{self.\_size}: count of integers is always accurate.\\

\begin{algorithmic}[1]
\Function{count}{$self$} 
    \State \Return $self.\_size$
\EndFunction
\end{algorithmic}
\vspace{5mm}

Our \texttt{sum()} function will return the sum of the stack, which is changed every time an \texttt{Integer} is inserted or removed from the stack, by the value of the \texttt{Integer}. \\

\textbf{Time complexity}: This operation runs in $O(1)$ time as it returns a pre-computed value.\\

\textbf{Correctness}: This operation returns the value of \texttt{self.\_sum} which is increased or decreased by the value of the Integer being pushed or popped in and out of the stack, respectfully. The invariant \texttt{self.\_sum} will always track the summed values of all Integers in the stack, hence the sum of integers is always accurate.\\

\begin{algorithmic}[1]
\Function{sum}{$self$} 
    \State \Return $self.\_sum$
\EndFunction
\end{algorithmic}
\vspace{5mm}

Our \texttt{average()} function calculates and returns the average of all elements in the stack. \\

\textbf{Time complexity}: This operation runs in $O(1)$ time as it only returns the quotient of two stored values. \texttt{self.\_sum} and \texttt{self.\_size} are maintained as aggregate values and do not require iteration over stack elements, which allows \texttt{average()} to run in constant time.\\

\textbf{Correctness}: This operations calculates the average based on \texttt{self.\_sum} and \texttt{self.\_size} which are both kept up to date with each push or pop operation. This calculation is based on the mathematical definition of the average, and the sum and size are always maintained accurately. It also checks to see if the list is empty to avoid runtime errors. Hence the calculated average sum is always accurate.\\

\begin{algorithmic}[1]
\Function{Average}{}
    \If{$self.\_size = 0$}
        \State \Return{$0$}
    \Else
        \State \Return{$self.\_sum / self.\_size$}
    \EndIf
\EndFunction
\end{algorithmic}

\newpage
\problem{3}  As input we are given a \textit{sorted} array $B$ containing $n$ positive integers, together with an integer $m$. The aim is to compute how many indices $i$ and $j$ (with $i < j$) there are such that the sum of the $i\text{th}$ and the $j\text{th}$ element of $B$ is at least $m$, i.e., $A[i] + A[j] >= m$. For full marks, your algorithm needs to run in $O(n)$ time.

Example:
$B = [1, 4, 4, 6], m = 7 \rightarrow \text{ return } 4$ \\

\textit{a) Design an algorithm that solves the problem.}

\begin{algorithmic}[1]
\Function{number\_of\_indices}{$B, m$}
    \State $n \gets \Call{Length}{B}$
    \State $count \gets 0$
    \State $i \gets 0$
    \State $j \gets n - 1$
    \While{$i < j$}
        \If{$B[i] + B[j] \geq m$}
            \State $count \gets count + (j - i)$
            \State $j \gets j - 1$
        \Else
            \State $i \gets i + 1$
        \EndIf
    \EndWhile
    \State \Return $count$
\EndFunction
\end{algorithmic}

The algorithm number \texttt{number\_of\_indices(array B, integer m)} takes advantage of the properties of the sorted list stored in $B$. Two pointers, $i$ and $j$ sum up two values, starting from the start (index $0$) and end (index $n-1$) of the list. We iterate through the array:

\begin{itemize}
    \item If $B[i]$ + $B[j] \ge m$: then these two values add up to at least $m$. Hence this pair, $(i, j)$ satisfy our condition, and we can add this pair of indices to our counter. But, because the list is sorted, this also means that, for every $i'$ where $i < i' < j$, the pair $(i', j)$ is also valid, and will sum to a value greater than $m$, so we add these pairs too. Hence we increment our counter "\texttt{count}" by the unique pairs of indices $(i', j)$ where $i \le i' < j$. which is $j-i$ indices. Then, to avoid double counting pairs with $j$, we decrement $j$ by $1$ to see if any smaller values of $j$ can add up with other values in the list to sum to equal or greater than $m$.
    \item If $B[i]$ + $B[j] \text{ NOT} \ge m \text{, i.e., }B[i]$ + $B[j] < m$: then these two values do not add up to at least $m$. We leave the pair counter alone and increment the $i$ pointer up to continue our analysis with a larger value.
    \item The loop ends when $i = j$, as we have now checked every possible pair in our list.
\end{itemize}

\textit{b) Argue the correctness of your algorithm.}

We can use loop invariants to argue the correctness of the two-pointer algorithm.  We want to check that, before and after each iteration of the loop:

\begin{enumerate}
    \item All pairs $(i', j')$, where $0 \le i' < i$ and $j < j' \le len(B) - 1$ have been considered, and that if those pairs summed to a value greater than or equal to $m$, that they have been counted.
    \item If $B[i] + B[j] \ge m$, then $B[i'] + B[j]$ for $i'$ where $i \le i' < j$ will also be greater than $m$, as $B$ is sorted in non-decreasing order, and thus $B[i'] \ge B[i]$. Hence all pairs $(i', j)$, where $i < i' < j$ would also have a sum greater than or equal to $m$, and should be counted too.
    \item All pairs $(i', j)$ where $i' < i$ and where $B[i] + B[j] < m$ are ignored and not added to the count, as $B[i'] < B[i]$.
    \item The count accurately reflects the number of valid pairs.
\end{enumerate}

\textbf{Initialisation}: $i = 0$ and $j = len(B) - 1$. There are no previously counted pairs. \\

\textbf{Iteration}: We check to see if the current pair add up to $m$: $B[i] + B[j] \ge m$. \\

If $B[i] + B[j] \ge m$:
\begin{itemize}
    \item The current pair $(i, j)$ is valid since their sum is at least $m$. 
    \item For every $i'$ where $i \le i' < j$, the pair $(i', j)$ is also valid, as $B[i']$ is greater than or equal to $B[i]$ due to the sorted list.
    \item The total number of these valid pairs is $(j-i)$, as there are $(j-i)$ other unique pairs of $(i', j) \ge m$ which we add to our counter. 
    \item Hence we have found all unique values of $i'$, which, when paired with $j$, $B[i'] + B[j] \ge m$. They have been considered and counted.
    \item Therefore, we must now decrement $j$ by $1$, to consider the next largest value of $j$ (as the list is sorted) to avoid double counting and to consider other possible valid pairs.
\end{itemize}

If $B[i] + B[j] < m$:
\begin{itemize}
    \item The current pair $(i, j)$ is not valid since their sum is less than $m$. We do not count this.
    \item For every $i'$ where $i' \le i$, the pair $(i', j)$ is also \textbf{not} valid, as $B[i']$ is less than or equal to $B[i]$ due to the sorted list. Hence, we do not count these pairs either.
    \item In this case, there is no other $j$ that would add up to $i$, where its sum would be $\ge m$, as for every $j'$ where $i< j' \le j$, $B[i] + B[j']$ would be smaller than $B[i] + B[j]$. Hence, there are no possible pairs with this current $i$. Therefore, we need to increment $i$, to find other possible pairs whose sum is $\ge m$.
    \item At the end, we do not add to our pair counter. This maintains the accuracy of our loop invariant.
\end{itemize}

\textbf{Termination}: The loop terminates when $i = j$, which means that are no more unique pairs to consider. The algorithm has exhaustively considered all possible pairs without redundancy or omission. The count reflects all valid pairs, as determined by the loop invariant that has been maintained. 

Moreover, if there are no pairs, count will remain as $0$, and the algorithm will return  $0$. When the array is empty, a simple \texttt{if-statement} at the beginning of the function can catch this and return $0$.\\


\textit{c) Analyse the running time of your algorithm.}

The algorithm begins by creating and storing variables for the size of the array, set to \texttt{n}, a counter for valid pairs \texttt{count}, and an \texttt{i} and \texttt{j} pointer. These are run in constant time $O(1)$. Regardless of the number of integers in the array, the number of initialisation variables remain the same. This also then takes constant space.\\

The bulk of this algorithm comprises of the \texttt{while} loop which runs while $i < j$. Inside the \texttt{while} loop, \texttt{line 7} \texttt{if B[i] + B[j] >= m:} performs a constant-time check to see if $B[i] + B[j]$ is greater than $m$. Comparisons are run in constant time. If it is larger, then the \texttt{count} variable is updated in constant time, where arithmetic operations are run in constant time. Decrementing \texttt{j} by 1 in \texttt{line 9} is also run in constant time $O(1)$. Else, if the sum is less than $m$, then \texttt{i} is incrementing which takes place in constant $O(1)$ time. \\

The focus is the while loop. By nature, $i$ or $j$ can move across the array at most $n$ times in total, where $n$ is the length of the array. 

\begin{itemize}
    \item $i$ can increment from $(0, n-1)$, which is at most $n-1$ increments.
    \item $j$ can decrement from $(n-1, 0)$ which is at most $n-1$ decrements.
\end{itemize}

Whilst each pointer can move $n-1$ times at most, the while loop terminates when $i = j$. Given that $i$ and $j$ are at the start and end of the array and move toward each other, through increments or decrements, the loop will run until they meet, which can happen \textbf{at most}, after $n-1$ iterations. Given a number of integers $n$ in the array, there is no significantly distinguishable best and worst case. If there are absolutely no pairs, $i$ would have to increment $n-1$ times, until it reaches $j$, in which case the loop terminates in $O(n)$ time. If the sum of the elements at the initial indices $i, j$ are already greater than or equal to $m$, then the algorithm will still decrement $j$ in each iteration $n-1$ times until it meets with $i$, in which case the loop terminates in $O(n)$ time.\\

Even if we consider iterating through the last half of the list, where $i = \frac{n}{2}$, each pointer can increment/decrement at most $n-1 - \frac{n}{2}$, which runs in $\Omega(n)$ time. Hence, the asymptotically tight running time of this algorithm is $\Theta(n)$.

\end{document}